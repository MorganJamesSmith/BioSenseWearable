\subsection{Health and Safety}

Using the Health and Safety Guide posted on the course webpage, students will 
use this section to explain how they addressed the issues of safety and health 
in the system that they built for their project.

\subsection{Engineering Professionalism}

Using their course experience of ECOR 4995 Professional Practice, students 
should demonstrate how their professional responsibilities were met by the 
goals of their project and/or during the performance of their project.

\subsection{Project Management}

One of the goals of the engineering project is real experience in working on a 
long-term team project. Students should explain what project management 
techniques or processes were used to coordinate, manage and perform their project.

\subsection{Justification of Suitability for Degree Program}

All team members are studying Computer Systems Engineering.  This program
teaches hardware, software, and the intersection between them.  This gave us a
strong foundation for developing our project as an embedded system.  We
applied what we have learned in ELEC and SYSC courses related to embedded
processors to develop our own hardware platform.  This required us to
exercise our electronic circuit analysis skills using concepts like Ohms law or
Kirchhoff’s voltage and current laws to make sure our electronics work
properly.  In order to connect all of our sensors and interfaces to the
embedded processor we used what we have learned about microcontroller
peripherals such as UART, I2C, SPI, GPIO, and timer modules.  In addition to 
creating our own hardware, we wrote software that runs on the embedded processor
to control the device.  The C programming language was used to do this, and we 
applied much of the knowledge and techniques learned from several SYSC courses.  
This includes concepts like interrupts, managing multiple tasks and events, 
using the hardware features of the embedded processor and more.  Some of the
non-technical courses included in the Computer Systems Engineering program have 
also proved useful for managing the project and writing reports.  Since we 
have applied large portions of the knowledge gained from our studies to
out project we believe the project is relevant to what we have studied and 
relevant to our degree program.

\subsection{Individual Contributions}

This section should carefully itemize the individual contributions of each team 
member. Project contributions should identify which components of work were done 
by each individual. Report contributions should list the author of each major 
section of this report.

\subsubsection{Project Contributions}

Contributions to the project

\subsubsection{Report Contributions}

Contributions to this report only