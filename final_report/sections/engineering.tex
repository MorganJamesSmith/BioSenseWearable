\subsection{Health and Safety}
% {\color{red}
% Using the Health and Safety Guide posted on the course webpage, students will 
% use this section to explain how they addressed the issues of safety and health 
% in the system that they built for their project.
% }
Due to Covid restrictions all electrical work was performed without supervision.  
Since all electrical work uses USB power of 5V no risk of serious injury was possible.
To mitigate risk of shock, it was ensured that work area is clear and that there are no live wires exposed while the board is connected to the power source.
All soldering was performed within reasonable safety guidelines.

\subsection{Engineering Professionalism}
% {\color{red}
% Using their course experience of ECOR 4995 Professional Practice, students 
% should demonstrate how their professional responsibilities were met by the 
% goals of their project and/or during the performance of their project.
% }
% Several aspects of the project were affected by engineering practices
In order to ensure user privacy is maintained, the data collected by the device is only stored locally and
can only be offloaded by USB (physical access) or over bluetooth using standard bluetooth encryption.
The device is enclosed in a way that ensures that electrical shock is not possible under normal operation.
The bluetooth module we are using is industry canada certified to be safe to use.

\subsection{Project Management}
% {\color{red}
% One of the goals of the engineering project is real experience in working on a 
% long-term team project. Students should explain what project management 
% techniques or processes were used to coordinate, manage and perform their project.
% }
The project primarily used the Agile development techniques where each team member had well defined tasks to work on.
Good communication over instant messaging and occasional meetings to discuss updates and re-evaluating task assignment were used to distribute work.
Though we had created a timeline, due to underestimating the amount of time certain tasks would take to accomplish, the timeline had to be pushed back.
Working with a group and deadlines enabled the development of time management skills.
% \begin{itemize}
%    \item regular meeting to discuss open design decisions
%    \item clear definition of roles, assigning tasks to each group member each meeting to ensure everyone had things to work on.
%    \item good communication amounst team members over instant messaging and bi-weekly meetings
%    \item managing hardware - who would order what and how the hardware would be distributed.
%    \item though we had created a timeline, due to underestimating the amount of time certain tasks would take to accomplish, the timeline had to be pushed back
%    \item working with a group and deadlines enabled the development of time management skills
% \end{itemize}
\subsection{Justification of Suitability for Degree Program}

All team members are studying Computer Systems Engineering.  This program
teaches hardware, software, and the intersection between the two topics.  This gave us a
strong foundation for developing our project as an embedded system.  We
applied what we have learned in ELEC and SYSC courses related to embedded
processors to develop our own hardware platform.  This required us to
exercise our electronic circuit analysis skills using concepts like Ohms law or
Kirchhoff’s voltage and current laws to make sure our electronics work
properly.  In order to connect sensors and interfaces to the
embedded processor we used what we have learned about microcontroller
peripherals such as UART, I2C, SPI, GPIO, and timer modules.  In addition to 
creating our own hardware, we wrote software in the C programming language to run on the embedded processor
to control the device,including concepts like interrupts, managing multiple tasks and events, 
using the hardware features of the embedded processor and more.  Some of the
non-technical courses, such as CCDP 2100, included in the Computer Systems Engineering program have 
also provided opportunity to practice project management and writing skills. 

Therefore, due to the previous courses taken by members of the group, we feel as though we can apply our theoretical knowledge to a practical application more efficiently

\subsection{Individual Contributions}
In this section, the contributions of each group member to the project are
presented. The contributions are divided into those related to the project 
itself, and secondly those related to the writing of this report.

\subsubsection{Project Contributions}
The list below details the contributions of each group member towards the
implementation and testing of the project. In cases where components were 
shared by group members, percentages are given.

\begin{itemize}
    \item Hardware component research and selection - Jason 50\%, Sam 50\%
    \item Electronic prototyping - Jason
    \item Schematic capture - Jason 80\%, Sam 20\%
    \item PCB layout - Jason
    \item PCB assembly - Sam
    \item Purchasing - Sam
    \item ...
    \item Command line interface - Sam
    \item USB driver - Sam
    \item Bluetooth driver - Jason
    \item Bluetooth interface for companion software - Jason
    \item ...
    \item Enclosure design - Morgan
    \item Enclosure fabrication - Morgan
    \item ...
\end{itemize}

\subsubsection{Report Contributions}
The list below details the contributions of each group member towards this final
report. In cases where sections were shared by group members, percentages are
given.

\begin{itemize}
    \item 2.4 Justification of Suitability for Degree Program - Jason
    \item 3 State of the Art - Jason
    \item ...
    \item 4.1 Component Selection - Sam 80\%, Jason 20\%
    \item 4.2 Schematic - Jason
    \item 4.3 PCB Layout - Jason
    \item ...
    \item 5.1.2 Bluetooth - Jason
    \item ...
    \item 8.1 Hardware Challenges - Jason ?\%, Sam ?\%
    \item 8.2 Software Challenges - Sam ?\%, Jason ?\%
    \item ...
    \item Appendix A - Jason
    \item ...
\end{itemize}