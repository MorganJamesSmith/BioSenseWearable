\subsection{Health and Safety}
% {\color{red}
% Using the Health and Safety Guide posted on the course webpage, students will 
% use this section to explain how they addressed the issues of safety and health 
% in the system that they built for their project.
% }
Due to Covid restrictions all electrical work was performed without supervision.  
Since all electrical work uses USB power of 5V no risk of serious injury was possible.
To mitigate risk of shock, it was ensured that work area is clear and that there are no live wires exposed while the board is connected to the power source.
All soldering was performed within reasonable safety guidelines.

\subsection{Engineering Professionalism}
% {\color{red}
% Using their course experience of ECOR 4995 Professional Practice, students 
% should demonstrate how their professional responsibilities were met by the 
% goals of their project and/or during the performance of their project.
% }
% Several aspects of the project were affected by engineering practices
In order to ensure user privacy is maintained, the data collected by the device is only stored locally and
can only be offloaded by USB or SD card (physical access) or over bluetooth using standard bluetooth encryption.
The device is enclosed in a way that ensures that electrical shock is not possible under normal operation.
The bluetooth module we are using is industry canada certified to be safe to use.

\subsection{Project Management}
% {\color{red}
% One of the goals of the engineering project is real experience in working on a 
% long-term team project. Students should explain what project management 
% techniques or processes were used to coordinate, manage and perform their project.
% }
Each member of the team was, due to individual passions, interested in different
aspects of the project, thus division of tasks was organic and did not require a leader or dictating division of work.
Good communication over instant messaging and occasional meetings to discuss updates and re-evaluating
task assignment ensured fair distribution and completion of work.
Though we had created a timeline,
due to underestimating the amount of time certain tasks would take to
accomplish, the timeline had to be pushed back. Working with a group and
deadlines enabled the development of time management skills.


\subsection{Justification of Suitability for Degree Program}

All team members are studying Computer Systems Engineering.  This program
teaches hardware, software, and the intersection between them.  This gave us a
strong foundation for developing our project as an embedded system.  We
applied what we have learned in ELEC and SYSC courses related to embedded
processors to develop our own hardware platform.  This required us to
exercise our electronic circuit analysis skills using concepts like Ohms law or
Kirchhoff’s voltage and current laws to make sure our electronics work
properly.  In order to connect all of our sensors and interfaces to the
embedded processor we used what we have learned about microcontroller
peripherals such as UART, I2C, SPI, GPIO, and timer modules.  In addition to 
creating our own hardware, we wrote software that runs on the embedded processor
to control the device.  The C programming language was used to do this, and we 
applied much of the knowledge and techniques learned from several SYSC courses.  
This includes concepts like interrupts, managing multiple tasks and events, 
using the hardware features of the embedded processor and more.  Some of the
non-technical courses included in the Computer Systems Engineering program have 
also proved useful for managing the project and writing reports. 
% this final sentence is weird.
Since we 
have applied large portions of the knowledge gained from our studies to
our project we believe the project is relevant to what we have studied and 
relevant to our degree program.
% due to the previous courses taken by members of the group, we feel as though we can apply our theoretical knowledge to a practical application more efficiently

\subsection{Individual Contributions}
In this section, the contributions of each group member to the project are
presented. The contributions are divided into those related to the project 
itself, and secondly those related to the writing of this report.

\subsubsection{Project Contributions}
The list below details the contributions of each group member towards the
implementation and testing of the project. In cases where components were 
shared by group members, percentages are given.

\begin{itemize}
    \item Hardware component research and selection - Jason 50\%, Sam 50\%
    \item Electronic prototyping - Jason
    \item Schematic capture - Jason 80\%, Sam 20\%
    \item PCB layout - Jason
    \item PCB assembly - Sam
    \item Hardware Debugging - Sam
    \item Bill of materials and Purchasing - Sam
    \item Command line interface - Sam
    \item USB driver - Sam
    \item Analog interface code - Sam
    \item ICM 20948 IMU driver - Sam
    \item Data logging and file systems - Sam
    \item Bluetooth driver - Jason
    \item Bluetooth interface for companion software - Jason 60\%, Tadhg 40\%
    \item Companion app CLI interactions and usb connection - Tadhg
    \item log file parsing and plotting - Tadhg
    \item Companion app automated testing - Tadhg
    \item Enclosure design - Morgan
    \item Enclosure fabrication - Morgan
    \item MAX86150 PPG/ECG driver - Morgan
    \item Guix package - Morgan
    \item SI1133 UV light sensor driver - Tadhg
\end{itemize}

\subsubsection{Report Contributions}
The list below details the contributions of each group member towards this final
report. In cases where sections were shared by group members, percentages are
given.

\begin{itemize}
    \item 0 Abstract - Morgan
    \item 1.1 Motivation - Morgan 50\%, Tadhg 50\%
    \item 1.2 Problem Statement - Jason
    \item 1.3 Solution - Tadhg
    \item 1.4 Accomplishments - Morgan
    \item 1.5 Report Structure - Tadhg
    \item 2.1 Health and Safety - Tadhg
    \item 2.2 Engineering Professionalism - Tadhg
    \item 2.3 Project Management - Tadhg
    \item 2.4 Justification of Suitability for Degree Program - Jason
    \item 3 State of the Art - Jason
    \item 4.1 Component Selection - Sam 80\%, Jason 20\%
    \item 4.2 Schematic - Jason
    \item 4.3 PCB Layout - Jason
    \item 4.4 Enclosure - Morgan
    \item 5 Software Implementation Description - Sam
    \item 5.4.1 USB - Sam
    \item 5.4.2 Bluetooth - Jason
    \item 5.4.3 Heart Rate - Morgan
    \item 6.1 Integration Testing - Sam
    \item 6.2 Performance Measurements - Sam
    \item 6.3 Power Measurements - Jason
    \item 6.4 Companion App Testing - Tadhg
    \item 6.5 Physical Dimensions - Morgan
    \item 6.6 Cost - Sam
    \item 7.1.1 Power and Battery Design Challenges - Jason
    \item 7.1.2 Assembly and Parts Availability Challenges - Sam
    \item 7.2.1 Toolchain and nRF5 SDK Challenges - Sam
    \item 7.2.2 USB and Bluetooth Integration Challenges - Jason
    \item 8 Timeline - Morgan
    \item 9 Conclusion - Tadhg
\end{itemize}
