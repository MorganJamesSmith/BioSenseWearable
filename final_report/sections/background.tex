Traditionally, people only seek treatment for ailments after they start
experiencing symptoms; however, it is often desirable to take action before
symptoms appear.  One solution to catch illness earlier could be to monitor a
person's physiological state using various sensors and take action when the
state falls outside of a known healthy state.  However, the accepted healthy
range for a population could be significantly larger than the accepted healthy
range for the individual \cite{Wearable-tracking2017}.

Wearables are capable of revealing personal daily physiological cycles with
respect to skin temperature, and heart rate with high accuracy
\cite{Wearable-tracking2017}. By comparing a person's current physiological
state with the expected state, it is possible to predict illness with greater
sensitivity than a self diagnosis. For example, a study was able to detect a
period of inflammation (confirmed by a medical high-sensitivity C-reactive
protein test) using a participant's heart rate even though the participant
themselves experienced no symptoms \cite{Wearable-tracking2017}.  The same
study even helped a participant get treatment of Lyme disease before the
``Bull's eye'' rash appeared (a common symptom during the early stages of Lyme
disease) by detecting an abnormality in their SpO\textsubscript{2} levels.

Wearable data can also be used to detect group illness rates.  By using heart
rate data and sleep data from fitbit devices and cross referencing that data
with the Centers for Disease Control and Prevention's (CDC) Influenze Like
Illness (ILI) data, a study was able to predict illness 6.3\% to 32.9\% better
than previous methods \cite{Radin2020}.

Wearables can also detect more immediate dangers to someone's health like a
heart attack \cite{heart-attack}, a fall \cite{Khojasteh_2018}, or dangerously
low or high heart rates.  This information could be used to alert emergency
services to save lives.

In summary, by creating a personal physiological profile and constantly
comparing the user's current physiological state to it, illness can be detected
early.  Our project focus is to create a device capable of measuring enough of
the user's current physiological state to predict illness.  This is done in the
hope that it could be used to slow the spread of illness's and, to speed
recovery from illness's by starting treatment earlier.
