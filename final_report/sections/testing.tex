\subsection{Integration Testing}

\subsection{Performance Measurements}

We made a number of timing measurements for various properties of our firmware
in order to get an understanding of whether our system is able to consistently
meet its timing requirements.

Measurements where taken using a hardware timer configured to run at 8 MHz. Code
to start the timer and code to stop the timer and store the current count was
added in places of interest. For each of the measured quantities 4096 samples
were recorded.

Table \ref{table:timing-results} shows the results of our timing measurements.
For each result the mean value of all recorded samples is given as well as the
standard deviation, and maximum recorded value. For many of the recorded values
almost all of the samples were within a very small range with only a handful
of outliers, because of this a 98th percentile of deviation value is given for
each measured quantity which gives the offset from the mean in which 98\% of the
samples fall.

Later in this section some of the more illuminating measurements are described
in detail.

\begin{longtable}{>{\centering\arraybackslash}m{3.1cm}|
                >{\centering\arraybackslash}m{2.2cm}|
                >{\centering\arraybackslash}m{3.5cm}|
                >{\centering\arraybackslash}m{1.8cm}|
                >{\centering\arraybackslash}m{2.2cm}}
\toprule
Test & Mean Time & 98th Percentile of Deviation & Standard Deviation &
Maximum \\
\midrule
Time between system ticks (millis)  & 998.868 microseconds & 98\% of results
within 250 nanoseconds of the mean & 1.185e-5 & 1.077 milliseconds \\
Time spent executing body of main loop & 5.625 microseconds & 98\% of results
within 16 microseconds of the mean & 6.327e-4 & 35.071 milliseconds \\
Time spend sleeping per iteration of main loop & 931.475 microseconds & 98\% of
results are within 872 microseconds of the mean & 2.14e-4 & 994.624 microseconds
\\
Time to perform a filesystem write operation & 16.722 milliseconds & 98\% of
results are within 31.484 microseconds of the mean & 1.362e-4 & 270.318
milliseconds \\
Time between data-logging write operations & 1.196 seconds & 98\% of results are
within 32 milliseconds of the mean & 9.156e-3 & 1.378 seconds \\
Time to perform an ADC sweep (tigger to data logging call) & 154.842
microseconds & 98\% of results are within 9 microseconds of the mean & 3.359e-4
& 14 milliseconds \\
Time between ADC interrupt and handling of ADC results in main loop & 17.435
microseconds & 98\% of results are within 12.934 microseconds of the mean &
4.171e-4 & 15.854 milliseconds \\
Time to retrieve IMU sample & 42.009 microseconds & 98\% of results are within
1.249 microseconds of the mean & 2.801e-5 & 1 millisecond \\
Time between IMU samples & 101.059 milliseconds & 98\% of results are within
60.184 microseconds of the mean & 3.005e-3 & 288.711 milliseconds \\
\bottomrule

\caption{Timing Results}
\label{table:timing-results}
\end{longtable}

\subsubsection{System Tick}

The timing results show that our system tick, which is use to maintain a count
of milliseconds ('millis'), is reasonably accurate and precise.

While the average period between SysTick interrupts is slightly less then the 1
millisecond target, the count is very precise. All of the measured samples were
within 750 microseconds of the mean value and 99\% of them where within 375
nanoseconds of the mean which indicates that there is very little jitter in our
millis value.

It is also important to note that for those samples which fell further from the
mean we generally see a delayed interrupt which is immediately followed by a
shorter interrupt period, so over time the average millisecond period remains
accurate even when there is some jitter. This happens because the SysTick timer
is configured in free running mode where it is automatically cleared after each
period and continues counting without needing to be restarted in the SysTick
interrupt handler.

\subsubsection{Data-logging}

Writes to the SD card are the longest blocking operation that happens in the
main loop. On average, a write operation takes about 16.7 milliseconds, but this
is far from the whole picture.

Figure \ref{fig:write-op-hist} is a histogram showing the distribution of write
operation lengths. To make the histogram legible the data has been trimmed to
remove outliers. Only samples within the 98th percentile of deviation where
kept, leaving 4014 samples in the dataset.


\begin{figure}[!htb]
\centering

\begin{gnuplot}[terminal=pdf,terminaloptions=color]
set xlabel "Length of write operation (s)"
set ylabel "Number of occurrences"

Min = 0.00709125 # where binning starts
Max = 0.037765875 # where binning ends
n = 100 # the number of bins

width = (Max-Min)/n
bin(x) = width*(floor((x-Min)/width)+0.5) + Min

unset key
set style fill solid 0.5
set xrange [Min:Max]

%plot 'data/data_log_write_trimmed' using (bin($2)):(1.0) smooth freq with boxes
\end{gnuplot}

\caption{Distribution of Filesystem Write Operation Lengths}
\label{fig:write-op-hist}
\end{figure}

The majority of filesystem write operations take about 12.7 milliseconds, but
the distribution is multimodal and there are also a number of write operations
that take between 25 and 30 milliseconds and around 35 milliseconds. The large
number of writes that take around 12.7 milliseconds are single block writes, the
other groups of write times are 2 and 3 block write operations. This timing data
shows that most of the write operations to the SD card only need to update a
single block, but that occasionally it is necessary to update 2 or 3 blocks at
once.

\subsubsection{Main Loop}

The loop body is made up of a number of service functions which implement the
various software modules that make up the system. Ideally the main loop would
run frequently in order to provide a responsive system. This cooperative
multitasking scheme relies on each of the modules not to block for long periods
in the main loop.

For the most part the loop is very short. The vast majority of loop iterations
measured (more than 90\%) took about 5.5 microseconds. These would be loop
iterations where none of the service functions had any work to do.

Similarly, for about 90\% of loop iterations the processor spends approximately
1 millisecond sleeping. This indicates that interrupt source which wakes the
processor most frequently is the SysTick timer that is used to generate
interrupts every millisecond.

The fact that so many iterations are not performing any work is a sign that the
processor is being woken more often than it needs to be. It would likely we
worth investigating alternative ways of generating our system time value other
than with regular SysTick interrupts such as using a hardware RTC in order to
reduce how often the processor is woken.

\subsubsection{Sensor Sampling}

ADC sweeps were generally quite quick. In over 99\% of measurements the time
from when the sweep was triggered to when the results where processed was only
about 155 microseconds. The time from ADC interrupt to ADC results processing
in the main loop was less than 30 microseconds in 99\% of measurements.

There were however some outliers where ADC sweeps took significantly longer.
The longest delay between ADC interrupt and the results being processed in the
main loop was about 15.8 milliseconds. This time happens to correspond with the
delay for a single block write to the SD card and it is likely that in these
outlier cases the ADC interrupt occurred while a block write was in progress.
This implies that we could see delays of up to around 38 milliseconds in
processing ADC data, but this is not a large concern because our ADC sampling
rate is quite low.

For the IMU, the time required to read a sample from the IMU was measured. This
non-blocking operation is spread out across several iterations of the main loop.
The operation started, then later calls to the IMU service function check
whether it has completed.

In all but a handful of samples it took about 42 microseconds to read a sample
from the IMU. In the remaining samples it took slightly more than a millisecond.
These longer sample occur when the interrupt at the end of the SPI transaction
occurs while the CPU is running code in the main loop after the IMU service
function has been called. In these cases the processor will sleep again before
starting at the top of the main loop and reaching the IMU service function.

These slightly delayed IMU read operations are not a large concern to us, the
IMU sample rate is low enough that a millisecond delay will not cause us to
miss a sample and since the IMU read operation is non-blocking it does not delay
anything else happening in the main loop.

What is more concerning is the possibility of an IMU sample being delayed by
coinciding with an SD card write operation. While we did not see this in our
testing, it remains a theoretical possibility and could cause IMU samples to be
significantly delayed. It would likely be worth investigating ways to make the
SD card write operations non-blocking so that they do not interfere with other
services in the main loop.

\subsection{Power Measurements}

\subsection{Companion App Testing}




%We didn't do this stuff yet 
\iffalse

We plan to write automated or semi-automated tests for each component of our
project. Most of these tests will take the form of unit tests, test harnesses
that emulate adjacent modules or small test applications that exercise a single
module.

\subsubsection{Unit Tests}

We plan to build a simple unit testing framework that allows us to write unit
tests that can be compiled and run on a host system. These tests will target
individual functions and will aim to validate that they are correct.

It is unlikely that we will be able to achieve full test coverage due to time
constraints. We will therefore focus on unit testing areas of the code that
are difficult to write test harnesses for, such as sensor drivers and the SD
card driver.

\subsubsection{Sensor Driver Testing}

In addition to unit tests, we will test the sensor drivers with semi-automated
test applications that run on the microcontroller. These tests will provide
feedback over the USB-CDC interface and will interact with the actual sensor
hardware.  Our sensor tests will aim to exercise all of the functionality of the
sensors that our application code makes use of.

These tests will be in the form of small console applications that will take
a series of sensor readings and display them. These readings will be manually
verified, since automatic verification of output that is based on real world
inputs is difficult. Due to the difficulty in writing automated tests for the
sensor drivers, these drivers will be an area of focus for unit testing.

\subsubsection{Bluetooth Stack Testing}

Our Bluetooth related firmware components will be developed alongside the
companion software. As they are developed together we will test them together,
using each to verify the correct operation of the other. Much of this testing
will be performed manually as we anticipate that tests that require interaction
between our device and the companion software will be more difficult to
automate.

\subsubsection{SD Card Driver Testing}

We will test the SD card driver and filesystem software using test applications
that will run directly on the microcontroller. A test application that reads
and writes blocks on the SD card and a test application that is able to read
from and write to files will be developed in order to validate the functionality
of the SD card and filesystem code. These tests will be small console
applications that a user can interact with using the the USB-CDC interface.

\subsubsection{Data Logging Code Testing}

In order to test the data logging component of the project we will create a test
framework that runs on a desktop computer which will emulate the sensor input
and the filesystem components in order to ensure that the application layer
encryption and file packing design works as intended. This will allow the data
logging code to be developed and tested independently of the filesystem and SD
card drivers to allow for more parallel development. This test framework will
create snapshots of the outputted data for a set of input conditions and ensure
that any updates to the logging code either does not change the output to ensure
nothing is broken during development, or if there is an error in these snapshots
then the snapshots will be updated.

\fi
