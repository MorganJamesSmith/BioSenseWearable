\subsection{Integration Testing}

Through our system's command line interface several intergration tests can be
run. These simple tests print out information that has been collected by sensors
for manual verification. Our command line interface can also be used to inspect
the contents of the attached SD card to verify the operation of the data logging
system.

Our data logging system can also be verified using our companion application,
which is designed to inform the user of invalid input that has been recorded to
the SD card.

\subsection{Performance Measurements}

We made a number of timing measurements for various properties of our firmware
in order to get an understanding of whether our system is able to consistently
meet its timing requirements.

Measurements where taken using a hardware timer configured to run at 8 MHz. Code
to start the timer and code to stop the timer and store the current count was
added in places of interest. For each of the measured quantities 4096 samples
were recorded.

Table \ref{table:timing-results} shows the results of our timing measurements.
For each result the mean value of all recorded samples is given as well as the
standard deviation, and maximum recorded value. For many of the recorded values
almost all of the samples were within a very small range with only a handful
of outliers, because of this a 98th percentile of deviation value is given for
each measured quantity which gives the offset from the mean in which 98\% of the
samples fall.

Later in this section some of the more illuminating measurements are described
in detail.

\begin{longtable}{>{\centering\arraybackslash}m{3.1cm}|
                >{\centering\arraybackslash}m{2.2cm}|
                >{\centering\arraybackslash}m{3.5cm}|
                >{\centering\arraybackslash}m{1.8cm}|
                >{\centering\arraybackslash}m{2.2cm}}
\toprule
Test & Mean Time & 98th Percentile of Deviation & Standard Deviation &
Maximum \\
\midrule
Time between system ticks (millis)  & 998.868 microseconds & 98\% of results
within 250 nanoseconds of the mean & 1.185e-5 & 1.077 milliseconds \\
Time spent executing body of main loop & 5.625 microseconds & 98\% of results
within 16 microseconds of the mean & 6.327e-4 & 35.071 milliseconds \\
Time spend sleeping per iteration of main loop & 931.475 microseconds & 98\% of
results are within 872 microseconds of the mean & 2.14e-4 & 994.624 microseconds
\\
Time to perform a filesystem write operation & 16.722 milliseconds & 98\% of
results are within 31.484 microseconds of the mean & 1.362e-4 & 270.318
milliseconds \\
Time between data-logging write operations & 1.196 seconds & 98\% of results are
within 32 milliseconds of the mean & 9.156e-3 & 1.378 seconds \\
Time to perform an ADC sweep (tigger to data logging call) & 154.842
microseconds & 98\% of results are within 9 microseconds of the mean & 3.359e-4
& 14 milliseconds \\
Time between ADC interrupt and handling of ADC results in main loop & 17.435
microseconds & 98\% of results are within 12.934 microseconds of the mean &
4.171e-4 & 15.854 milliseconds \\
Time to retrieve IMU sample & 42.009 microseconds & 98\% of results are within
1.249 microseconds of the mean & 2.801e-5 & 1 millisecond \\
Time between IMU samples & 101.059 milliseconds & 98\% of results are within
60.184 microseconds of the mean & 3.005e-3 & 288.711 milliseconds \\
\bottomrule

\caption{Timing Results}
\label{table:timing-results}
\end{longtable}

\subsubsection{System Tick}

The timing results show that our system tick, which is use to maintain a count
of milliseconds ('millis'), is reasonably accurate and precise.

While the average period between SysTick interrupts is slightly less then the 1
millisecond target, the count is very precise. All of the measured samples were
within 750 microseconds of the mean value and 99\% of them where within 375
nanoseconds of the mean which indicates that there is very little jitter in our
millis value.

It is also important to note that for those samples which fell further from the
mean we generally see a delayed interrupt which is immediately followed by a
shorter interrupt period, so over time the average millisecond period remains
accurate even when there is some jitter. This happens because the SysTick timer
is configured in free running mode where it is automatically cleared after each
period and continues counting without needing to be restarted in the SysTick
interrupt handler.

\subsubsection{Data-logging}
\label{subsec:testing-data-logging}

Writes to the SD card are the longest blocking operation that happens in the
main loop. On average, a write operation takes about 16.7 milliseconds, but this
is far from the whole picture.

Figure \ref{fig:write-op-hist} is a histogram showing the distribution of write
operation lengths. To make the histogram legible the data has been trimmed to
remove outliers. Only samples within the 98th percentile of deviation where
kept, leaving 4014 samples in the dataset.


\begin{figure}[!htb]
\centering

\begin{gnuplot}[terminal=pdf,terminaloptions=color]
set xlabel "Length of write operation (s)"
set ylabel "Number of occurrences"

Min = 0.00709125 # where binning starts
Max = 0.037765875 # where binning ends
n = 100 # the number of bins

width = (Max-Min)/n
bin(x) = width*(floor((x-Min)/width)+0.5) + Min

unset key
set style fill solid 0.5
set xrange [Min:Max]

%plot 'data/data_log_write_trimmed' using (bin($2)):(1.0) smooth freq with boxes
\end{gnuplot}

\caption{Distribution of filesystem write operation lengths.}
\label{fig:write-op-hist}
\end{figure}

The majority of filesystem write operations take about 12.7 milliseconds, but
the distribution is multimodal and there are also a number of write operations
that take between 25 and 30 milliseconds and around 35 milliseconds. The large
number of writes that take around 12.7 milliseconds are single block writes, the
other groups of write times are 2 and 3 block write operations. This timing data
shows that most of the write operations to the SD card only need to update a
single block, but that occasionally it is necessary to update 2 or 3 blocks at
once.

\subsubsection{Main Loop}

The loop body is made up of a number of service functions which implement the
various software modules that make up the system. Ideally the main loop would
run frequently in order to provide a responsive system. This cooperative
multitasking scheme relies on each of the modules not to block for long periods
in the main loop.

For the most part the loop is very short. The vast majority of loop iterations
measured (more than 90\%) took about 5.5 microseconds. These would be loop
iterations where none of the service functions had any work to do.

Similarly, for about 90\% of loop iterations the processor spends approximately
1 millisecond sleeping. This indicates that interrupt source which wakes the
processor most frequently is the SysTick timer that is used to generate
interrupts every millisecond.

The fact that so many iterations are not performing any work is a sign that the
processor is being woken more often than it needs to be. It would likely we
worth investigating alternative ways of generating our system time value other
than with regular SysTick interrupts such as using a hardware RTC in order to
reduce how often the processor is woken.

\subsubsection{Sensor Sampling}

ADC sweeps were generally quite quick. In over 99\% of measurements the time
from when the sweep was triggered to when the results where processed was only
about 155 microseconds. The time from ADC interrupt to ADC results processing
in the main loop was less than 30 microseconds in 99\% of measurements.

There were however some outliers where ADC sweeps took significantly longer.
The longest delay between ADC interrupt and the results being processed in the
main loop was about 15.8 milliseconds. This time happens to correspond with the
delay for a single block write to the SD card and it is likely that in these
outlier cases the ADC interrupt occurred while a block write was in progress.
This implies that we could see delays of up to around 38 milliseconds in
processing ADC data, but this is not a large concern because our ADC sampling
rate is quite low.

For the IMU, the time required to read a sample from the IMU was measured. This
non-blocking operation is spread out across several iterations of the main loop.
The operation started, then later calls to the IMU service function check
whether it has completed.

In all but a handful of samples it took about 42 microseconds to read a sample
from the IMU. In the remaining samples it took slightly more than a millisecond.
These longer sample occur when the interrupt at the end of the SPI transaction
occurs while the CPU is running code in the main loop after the IMU service
function has been called. In these cases the processor will sleep again before
starting at the top of the main loop and reaching the IMU service function.

These slightly delayed IMU read operations are not a large concern to us, the
IMU sample rate is low enough that a millisecond delay will not cause us to
miss a sample and since the IMU read operation is non-blocking it does not delay
anything else happening in the main loop.

What is more concerning is the possibility of an IMU sample being delayed by
coinciding with an SD card write operation. While we did not see this in our
testing, it remains a theoretical possibility and could cause IMU samples to be
significantly delayed. It would likely be worth investigating ways to make the
SD card write operations non-blocking so that they do not interfere with other
services in the main loop.

\subsection{Power Measurements}

Electrical specifications for the wearable device are shown in Table \ref{table:elec-spec}.
This table includes expected battery charging times, battery life, and current 
consumption under various conditions. All current measurements are shown as an 
expected load on a fully charged battery. As the battery voltage decreases, the
current consumption will increase in order to maintain correct system voltages 
through a DC-DC conversion process.

The quiescent current refers to the current consumption of the device while 
it is turned off. The minimum operating current refers to the minimum current 
required to power up the device without it performing any useful functions. 
The typical operating current refers to the average current the device would 
consume when logging sensor data but not performing any communication. This is 
the condition the device would normally be in while a user is wearing it. The 
remaining current measurements are intermittent and apply only in specific 
situations, and only for the durations that those situations exist. For example, 
the indicator LED current is 15 mA only while the indicator LED is on and the 
colour is red. Several of these currents can be added together with the minimum 
operating current to estimate what the current consumption would be under 
different circumstances. If all three indicator LEDs are on, then the total 
device current can be estimated to be about 15mA (Red LED) + 10mA (Green LED) 
+ 10mA (Blue LED) + 3mA (Minimum Operating Current) = 38mA.

\begin{table*}[!htb]
\centering
\begin{tabular}{>{\centering\arraybackslash}m{6.0cm}|
                >{\centering\arraybackslash}m{4.6cm}|
                >{\centering\arraybackslash}m{1.0cm}|
                >{\centering\arraybackslash}m{1.2cm}}
\toprule
Parameter & Condition & Value & Unit \\
\midrule
Battery Capacity & & 400 & mAh \\
Battery Voltage & Nominal & 3.7 & V \\
Battery Charging Time & Slow charging & 5 & hours \\
Battery Charging Time & Fast charging & 1 & hours \\
Battery Life & Powered off & 6 & months \\
Battery Life & Normal operation & 2 & days \\
Quiescent Current & Powered off, battery voltage = 3.7V & 60 & {\textmu}A \\
Minimum Operating Current & Not collecting data, not communicating, battery voltage = 3.7V & 3 & mA \\
Typical Operating Current & Logging data, not communicating, battery voltage = 3.7V & 7.2 & mA \\
Indicator LED Current & Red only, battery voltage = 3.7V & 15 & mA \\
Indicator LED Current & Green only, battery voltage = 3.7V & 10 & mA \\
Indicator LED Current & Blue only, battery voltage = 3.7V & 10 & mA \\
Bluetooth Transmitting & Battery voltage = 3.7V & 5 & mA \\
Bluetooth Receiving & Battery voltage = 3.7V & 3 & mA \\
SD Card Current & Battery voltage = 3.7V & 2 & mA \\
Heart Rate Measurement Current & Battery voltage = 3.7V & 375 & {\textmu}A \\
UV Light Measurement Current & Battery voltage = 3.7V & 2 & mA \\
IMU 9-Axis Measurement Current & Battery voltage = 3.7V & 1.5 & mA \\
Temperature Measurement Current & Battery voltage = 3.7V & 625 & {\textmu}A \\
\bottomrule
\end{tabular}
\caption{Electrical Specifiactions}
\label{table:elec-spec}
\end{table*}

\subsection{Companion App Testing}

Most erroneous cases handled by the companion app are not fatal errors but warnings about skipped data. One example is that if an entry has an unrecognized type such as adding a new sensor then it is skipped and all other recognized entries are still parsed.

One notable case is that if an entry specifies it has a length of 28 bytes then
writes 10 bytes and loses power, upon reset the reset offsets file is updated with 
the length of the file and a reset command is written after the 10 bytes.  
In this case it is desirable to use the offset information in offsets to 
identify that the previous entry was not written in its entirety and discard it without allowing it to corrupt all subsequent entries.  Almost all tests for the companion app test scenarios like this where it is asserted that no fatal error was raised, the expected warning messages were given and all valid data is output as expected. As such there are no test cases to test "totally valid" data.

One case which correspond to fatal errors is when the offset specified by offsets file does not correspond to a reset command, in this case there is nothing reasonable the companion app can do to try to identify entry boundaries from the file so it gives up.  A second case which causes a fatal error is when the CLI returns the string "Failed to open file" when reading the file, as there is nothing that can be done if the SD card is corrupted and the file cannot be read, these are the only 2 fatal error cases that are tested for.

The CLI interaction is tested by creating a mock serial interface which follows the same protocols as the bluetooth or usb serial connection.  The bluetooth connection or USB serial connection are hard to write automated tests for so they are not covered by the test suite.

\subsection{Physical Dimensions}

\begin{longtable}{ c | c }
\toprule
Quantity & Value \\
\midrule
Weight of assembled circuit boards & 11 grams \\
Maximum assembled length & 6 cm \\
Maximum assembled width & 4.8 cm \\
Maximum assembled thickness & 1 cm \\
\bottomrule

\caption{Physical Dimentions}
\label{table:physical-dimentions}
\end{longtable}

