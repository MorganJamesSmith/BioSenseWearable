This section provides an overview of the problem this project aims to solve,
why it is important, how this project can be part of the solution, and finally 
a brief description of what progress was actually made towards the solving the 
problem. Subsequent sections of this report cover the hardware and software design 
of the project in detail, as well as discuss some of the challenges that
were faced in pursuing the solution outlined here. Information regarding the
tests performed on the final project are also detailed in this report, and a
series of appendices are included with visuals to show some interesting parts 
of the design and what the final product looks like.

\subsection{Background}

Traditionally, people only seek treatment for ailments after they start
experiencing symptoms; however, it is often desirable to take action before
symptoms appear.  One solution to catch illness earlier could be to monitor a
person's physiological state using various sensors and take action when the
state falls outside of a known healthy state.  However, the accepted healthy
range for a population could be significantly larger than the accepted healthy
range for the individual \cite{Wearable-tracking2017}.

Wearables are capable of revealing personal daily physiological cycles with
respect to skin temperature, and heart rate with high accuracy
\cite{Wearable-tracking2017}. By comparing a person's current physiological
state with the expected state, it is possible to predict illness with greater
sensitivity than a self diagnosis. For example, a study was able to detect a
period of inflammation (confirmed by a medical high-sensitivity C-reactive
protein test) using a participant's heart rate even though the participant
themselves experienced no symptoms \cite{Wearable-tracking2017}.  The same
study even helped a participant get treatment of Lyme disease before the
``Bull's eye'' rash appeared (a common symptom during the early stages of Lyme
disease) by detecting an abnormality in their SpO\textsubscript{2} levels.

Wearable data can also be used to detect group illness rates.  By using heart
rate data and sleep data from fitbit devices and cross referencing that data
with the Centers for Disease Control and Prevention's (CDC) Influenza Like
Illness (ILI) data, a study was able to predict illness 6.3\% to 32.9\% better
than previous methods \cite{Radin2020}.

Wearables can also detect more immediate dangers to someone's health like a
heart attack \cite{heart-attack}, a fall \cite{Khojasteh_2018}, or dangerously
low or high heart rates.  This information could be used to alert emergency
services to save lives.

\subsection{Motivation}

Despite all of the advances in medical technology, a cure for diseases and
allowing humans to live longer are ever present goals for scientists and 
doctors alike.  Often wide spread diseases we have not prepared for can occur in
society.  COVID-19 is a pandemic that
has, at the time of writing, claimed an estimated 2.5 million lives globally
\cite{johns-hopkins-corona-chan}, and continues to infect more people every
day.  In 2007 it was estimated that the total cost of occupational illnesses
among civilians in the United States was \$58 billion \cite{Leigh2011}.
Slowing the spread of disease could save lives and save money.

One method of limiting the spread of disease is early diagnosis.  This would
allow individuals who are aware they are ill to take measures to reduce the
spread of their illness, even before symptoms manifest.  This will be
particularly useful for healthcare providers who work with high risk
individuals, since identifying symptoms early has the highest efficiency for
protecting others.

By creating a personal physiological profile and constantly
comparing the user's current physiological state to it, illness can be detected
early.  The motivation for our project is to create a device capable of measuring 
enough of the user's current physiological state to predict illness.  This is done 
in the hope that it could be used to slow the spread of illness's and, to speed
recovery from illness's by starting treatment earlier.

\subsection{Problem Statement}

One way to facilitate early diagnosis is to measure biometrics on patients and
make clinical diagnoses based on these measurements.
A convenient and low cost method to collect biometric data is needed in order
to make the potential benefits of the technology available to a large number of
people. This information may then be used to detect abnormalities in the user’s 
physiology and could be used to aid in the early detection of illness.
The objective of this project is to create a wearable device that can measure a
variety of critical health metrics including skin temperature and heart rate.

\subsection{Solution}

The solution provided by this project is a wrist mounted wearable health 
monitoring device. The device provides a sensor platform that is suitable for 
early detection of disease. The included sensors can measure skin temperature, 
ambient temperature, heart rate, pulse oximetry, and motion.  The electronic
components fit into a comfortable, wearable 3D printed plastic enclosure.
The device is powered by a lithium polymer battery which can be recharged
through the USB-C connector.

The wearable device has limited user interfaces consisting of an indicator
LED for power, notifications, and debugging. It also features a power button
to turn the device on and off. The main interactions with the device occur 
through a companion software that connects to the device through Bluetooth or 
USB. It is limited to simply receiving and decoding data received from the 
wearable device, and exporting it in an analyzable format such as a csv or json 
file.

The device will store the collected sensor information to a micro SD card
in the device which is then read back when requested via the companion software.
The SD card will be replaceable, and the capacity of the card used will dictate
how long the device can record data before needing to off load to the companion
software.

The focus of this solution is on the collection of biometrics and making them
available for analysis. No attempt was made to perform medical analysis
or give clinical diagnosis from the collected biometrics. Our intention was to
design the system so that those applications could be integrated with or built
on top of the work we have done for this project.

\subsection{Accomplishments}
{\color{red}
What did we actually do?
}

%Should this be in a reflection or something? Not really part of a solution,
%it is a disclaimer.

%In order to reduce development cost and complexity, the hardware component of
%the project uses sensors that are readily available from online sources, and 
%that can be assembled on a PCB by hand relatively easily. As a result the 
%selection of sensors may be more limited or lower quality than what is possible 
%if options that are made to order, or only available in large quantities had been
%explored.  The use of parts that can be soldered to a PCB without paying for
%the services of an advanced manufacturing facility also resulted in the final 
%hardware design being larger than it would have been otherwise.
