\noindent
Despite all of the advances in medical technology, a cure for diseases and
allowing humans to live longer are ever present goals for scientists and 
doctors alike.  Often wide spread diseases we have not prepared for can occur in
society.  COVID-19 is a pandemic that
has, at the time of writing, claimed an estimated 1 million lives globally
\cite{johns-hopkins-corona-chan}, and continues to infect more people every
day.  In 2007 it was estimated that the total cost of occupational illnesses
among civilians in the United States was \$58 billion \cite{Leigh2011}.
Slowing the spread of disease could save lives and save money.


One method of limiting the spread of disease is early diagnosis.  This would
allow individuals who are aware they are ill to take measures to reduce the
spread of their illness, even before symptoms manifest.  This will be
particularly useful for healthcare providers who work with high risk
individuals, since identifying symptoms early has the highest efficiency for
protecting others.

One way to facilitate early diagnosis is to measure biometrics on patients and
make clinical diagnoses based on these measurements.
The objective of this project is to create a wearable device that can measure a
variety of critical health metrics including skin temperature and heart rate.
This information may then be used to detect abnormalities in the user’s 
physiology and could be used to aid in the early detection of illness.

The device will be in the form factor of a watch.  The top casing can be
removed to replace the internal SD card and AAA battery.  The battery can also
be charged using the device's USB type C port.  The device will store user
information in an encrypted format on the SD card and transmit the information
over a secure Bluetooth connection to companion software running on a separate
device.
