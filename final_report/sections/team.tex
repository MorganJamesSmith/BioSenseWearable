
All team members are studying Computer Systems Engineering.  This program
teaches hardware, software, and the intersection between them.  This gives us a
strong foundation for developing our project as an embedded system.  We will
apply what we have learned in ELEC and SYSC courses related to embedded
processors to develop our own hardware platform.  This will require us to
exercise our electronic circuit analysis skills using concepts like Ohms law or
Kirchhoff’s voltage and current laws to make sure our electronics work
properly.  In order to connect all of our sensors and interfaces to our
embedded processor we will be using what we have learned about microcontroller
peripherals such as UART, I2C, SPI, GPIO, and timer modules.  After creating
our own hardware, we will be programming the embedded processor on it.  The C
programming language will be used to do this, and we will apply much of the
knowledge learned in several SYSC courses.  This includes concepts like
interrupts, managing multiple tasks and events, using the hardware features of
the embedded processor and more.  Each team member also brings some unique
skills, knowledge, and experience integral to the project that they have gained
outside school through work, hobbies, or extracurricular activities. These
individual areas of experience will be mentioned in their respective sections
below, in addition to their role in the project.


% Project Tasks:

% Sam:
%     - Schematic capture and PCB layout
%     - Debugging interface
%     - Sensor Drivers
% Jason:
%     - Schematic capture and PCB layout
%     - Bluetooth Stack
% Morgan:
%     - Mechanical design
%     - SPI SD Card driver and filesystem driver
%     - Sensor Drivers
% Tadhg:
%     - Companion software
%     - Database/data storage code
%     - Sensor Drivers


\subsection{Sam}

The tasks that I will be working on for this project include schematic capture
and PCB design, a USB-CDC stack for debugging, writing sensor drivers, firmware
testing and PCB assembly.

I have a relatively large amount of experience experience working with embedded
software and hardware from personal projects, co-op work experience and the
CU InSpace design team at Carleton.  I believe that this experience has
positioned me quite well to work on this project.  In a number of personal
projects and with CU InSpace I have designed, implemented, tested and documented
a number of embedded systems that are not too dissimilar from what we plan to
build.

My experience includes designing PCBs around microcontrollers, sensors and RF
components, assembling those boards and testing them.  This experience should be
valuable in contributing to our hardware schematic design, PCB layout and to
hardware testing.

I am also well prepared to work on the USB interface for our project as I have
implemented a very similar system for CU InSpace in the past.  As part of the
CU InSpace project I have also implemented a number of drivers for sensors and
other peripherals, which should give me a solid background for approaching the
sensor drivers that will need to be written for this project.

Finally, I have a reasonably strong background in unit testing of low level
software through co-op experience which I was later able to apply to CU InSpace
as well.  I hope to build on this experience with unit testing and regression
testing in order to build test frameworks for this project.

\subsection{Tadhg}

% Stack-overflow
% ericsson
% debug and problem solve, also code review

% TODO TADHG: need to heavily work this paragraph
The thing I've always been good at is learning a bit of everything quickly and
giving helpful advice regarding coding to others.  I've been a active member of Stack Overflow
and have answered over 400 questions.  As well during my co-op at Ericsson I was
known in my team for giving extensive code reviews.  As such I will oversee pushed
code to ensure correctness and good conventions to ensure any issues that arise will
be easy to track down and any assumptions the program requires are asserted
within the program.

I am very knowledgeable with high level software applications which makes me a good
candidate to write the companion software which will extract the logged data and
produce basic visual graphs for testing and demonstrative purposes

While all of the members of the project have limited experience working with databases
I have experience designing and describing data mappings for complex data types,
therefore making me the better choice for handling the database and logging system.

In order to write drivers for sensors I will need to get accustomed to reading
data sheets.  I pick up on new skills quickly and am confident I will be able
to perform this task well.


% The skills I bring to this project are primarily software development and
% high level product design.

% With this skill set I am a good fit to be responsible
% for the data extraction interfaces and companion software.  I will also be handling
% the database structure to store the logs on the device.

\subsection{Jason}
	The main tasks I will be working on for this project are schematic 
	capture, PCB design, and Bluetooth communication on the wearable device.
	
	I believe I am well suited for the schematic capture and PCB design portions of
	this project as I have created several schematics and a few PCB designs in the 
	past.  This experience comes partially from self teaching at home and working on
	some hobby projects, but a significant portion was acquired through the COOP
	work experience program.  I was fortunate to have the opportunity to work
	with people who have decades of experience in schematic capture and board design,
	and were willing to let me try it on my own with their guidance.

	A natural extension of hardware design is hardware testing.  Once we have our
	hardware in hand, another task I will be helping with is ensuring the hardware
	works as it was designed.  This is another skill I had significant practice with during
	my COOP work terms and with my own projects at home.  I am confident that we
	will be able to find and fix any hardware problems we encounter in this project.
	
	Bluetooth communication is something I do not have specific experience with,
	but we are planning to use a vendor provided Bluetooth stack to advance the
	project quicker.  I have successfully used vendor provided USB stacks a couple
	of times in the past and I believe I will be able to learn from the documentation 
	and do the same for Bluetooth.

	Aside from the main tasks mentioned, I will likely spend the rest of my time on
	general C programming of the device firmware.  I am confident in my abilities to
	do this as I again have a decent amount of experience from work outside of
	school, and some of the courses taught in school.

\subsection{Morgan}

The tasks that I will be working on for this project include the mechanical
design and fabrication, SD card driver, sensor drivers and filesystem code.

I am suited for the mechanical design due to my experience in 3D printing and
3D modeling.  Previously I have modeled a fully functional Rubik's cube,
various 3D printer parts, and various fasteners using Fusion360 and OpenSCAD.
I intend to use OpenSCAD for all the mechanical design.

I am also well suited to writing embedded drivers as I have professional
experience.  During my co-op experiences I have written radio drivers
compatible with both MSP and ARM processors.  This experience will help me
develop the SD Card and sensor drivers.

I am confident in my ability to create the filesystem code as I regularly
manipulate Linux block devices.  This will give me the ability verify
filesystem operations made by the device using the fsck command.  I will also
be able to create scripts that will format the SD card and then place data on
it in a reproducible way for testing.
