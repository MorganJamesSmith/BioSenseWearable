Several challenges were faced during the course of this project with both the
hardware and software development. This section will describe some of the 
challenges faced, how they were overcome, and what trade-offs or compromises,
if any, had to be considered when finding a solution.

\subsection{Hardware Challenges}

\subsubsection{Power and Battery Design}

One of the major challenges that occurred with the hardware development early
on in the project was the power supply. The original plan as described in the 
proposal was to use a single rechargeable AAA nickel metal hydride (NiMH) 
battery. Designing a power supply to work with the low input voltage of 1.2V 
to 1.0V proved to be quite difficult due to a small selection of components that 
work at that voltage. The most difficult part was trying to implement a soft 
power on / off circuit as well as an automatic switchover circuit to shift the 
electrical load from the battery to USB power. The voltage drop across a diode 
for example makes the resulting signal nearly unusable to turn on a MOSFET. 
Diodes with a very low forward voltage drop and MOSFETS with very low gate 
threshold voltages are available for a higher price, but keeping cost down was 
also a very important goal. In the end a design was created, and since it was 
expected that it may present some problems, it was prototyped on a breadboard 
so that testing could be done before moving ahead with schematic capture and 
PCB design.

Testing showed that the design mostly worked, but had some flaws that were 
deemed unacceptable for this project. The first concern was that the power 
supply would not be able to supply enough current. It was capable of powering 
an nRF52840 development board with some additional resistive load, but it 
struggled to maintain a set voltage and it became apparent that it would be 
barely capable of powering the final design with several sensors. The second 
issue was with voltage regulator start up under load. The switching regulator 
was only able to start with a very small load on it, but once the output had 
reached the desired voltage more load could be placed on it without issue. When 
too much load was attached to the regulator before start up it would stay in an 
under-voltage lockout state indefinitely. This problem may or may not have 
presented itself in a real-world test with the final design, however the goal 
was to have a robust solution that was reliable and not leave things to chance 
with a potential race condition. The final problem was efficiency. With a light 
load the power supply performed in an acceptable fashion, but under heavier 
loads efficiency dropped off much more than expected. The datasheet for the 
regulator did not indicate that efficiency would change appreciably over the 
range of current being drawn for testing, but it did indicate that lower input 
voltages would cause a significant decrease in efficiency. It is believed that 
higher load caused the battery voltage to drop due to its series resistance 
which then caused the regulator to draw much more current to account for the 
higher load and decreased efficiency at the same time because of the decreased 
input voltage. Replacing the battery with a lab power supply set to constant 
voltage showed the expected efficiency under both light and heavy loads. For 
these three reasons it was decided that the original idea of using a single 
rechargeable AAA NiMH battery was not feasible. 

At this point the group had to decide between using two AAA NiMH batteries in 
series to increase the voltage, or starting over and using a LiPo battery. It 
was decided that two AAA batteries would be too large, so the switch to LiPo 
was made and the power supply was subsequently redesigned to reflect these 
changes. One of the design requirements in the proposal was to have the 
batteries removable so they could be swapped out with charged ones instead of 
waiting for the device to charge them. This will no longer be possible with 
the current design; however fast charging capability has now been added as a 
replacement.

\subsubsection{Assembly and Parts Availability}

One of the key challenges that we planned for and managed on the hardware side
of our project was the difficulty of assembling our circuit boards by hand. Due
to the sensors we wanted to use coming in packages designed for small, high
density printed circuit board designs and our desire to build a prototype that
was roughly wrist sized we ended up using a number of components that were not
particularly well suited to hand assembly.

For the most part our assembly still worked out well using a hot air reflow
technique, but we did encounter significant frustration and delay in soldering
our microcontroller module to the board. The module we used was not amenable to
the hot air reflow technique since the pads that needed to be soldered are
located below an RF shielding can. This means that it took a considerable amount
of time and heat to properly reflow the solder paste below the module and we had
inconstant results with no way of properly inspecting the module to ensure that
it had been soldered properly. In the end we simply ended up repeatedly removing
the module and then soldering it back on until we were lucky enough to have all
of the pins flow properly.

An issue that we did not foresee, but should have, was limited availability of
components. When originally designing our hardware we limited ourselves to
components that were readily available in small quantities from online
retailers and that were in stock in large quantities at the time. Between
finalizing our schematic design near the end of the fall semester and ordering
the components for our first copy of the board in the winter semester the IMU
and air quality sensors we had selected had gone out of stock at all of the
electronics retailers that we were familiar with. The IMU later became
available again, but we were never able to obtain the air quality sensor.

A few weeks later when we went to order parts for our second copy of the board
we found that even more components had become unavailable. In addition to the
air quality sensor, the microcontroller module, diodes, USB connector and UV
light sensor that we had chosen for our design were not available. We were
able to find a substitute microcontroller module, though the substitute requires
an external Bluetooth antenna and suitable replacement diodes, but we were
forced to leave the USB connector and UV light sensor unpopulated on the second
copy of our board.


\subsection{Software Challenges}

\subsubsection{Toolchain and nRF5 SDK}
\label{sec:toolchain-issues}

Our toolchain setup ended up being much more difficult than we had envisioned.
In our proposal we had only budgeted ourselves a couple days for this step,
assuming that our experience with other ARM microcontrollers would allow us to
get up and running with the nRF52840 quite quickly. In reality, this seemingly
simple task had required a significant investment of time which held up our
software progress early in the project considerably.

We are used the hardware abstraction layer provided by Nordic
Semiconductor for the nRF52840 as well as some of their libraries. While we had
a build system set up that allowed us to successfully compile and run bare metal
software without any of Nordic's nRF SDK components, actually compiling the SDK 
along with our own code proved to be significantly more challenging. We 
originally tried to adopt our custom Makefile to also compile the nRF52 SDK files 
we required, but ended up having to base our project layout off of one of the 
example projects provided with the nRF52 SDK. A surprising amount of work was 
required to get the Makefile and Segger Embedded Studio project from the example 
to work in our repository without having to place all of our code in the examples 
folder inside the nRF52 SDK's directory structure. The makefiles provided in the
nRF52 SDK are inflexible and make use of a large number of hard coded paths.

\subsubsection{USB and Bluetooth Integration}

One of the challenges faced during integration testing was a conflict between 
the USB and Bluetooth drivers. The two drivers were developed independently of
each other, and both worked on their own. When running the two drivers together
Bluetooth would function as expected but USB would fail to enumerate on the host
and therefore, not work at all. Oddly, this problem would not occur when the 
microcontroller was running in debug mode, it presented itself only in the 
normal run mode. This made the problem much more difficult to solve because
stepping though code, breakpoints, or any other debugging capabilities were not
available outside of debug mode. The alternative technique used was to turn on 
an LED when execution reached a certain point or some condition was met. 

In the end, the solution to the problem ended up being very easy to implement 
despite taking a significant amount of time to find. When Bluetooth is 
initialised, it enables what Nordic calls a Soft Device. This Soft Device runs 
the Bluetooth stack and also takes over control of some parts of the hardware 
in the process. If other components of the software need access to hardware 
controlled by the Soft Device, then they must make calls to the Soft Device to 
perform the necessary actions on their behalf. The USB driver relies on events 
from the hardware to inform it when it is attached to a host so it knows when 
to enumerate and begin communicating. These hardware events happen to be blocked 
by the Soft Device, so if the USB driver wants to receive them it must arrange 
this with the Soft Device after it has been enabled. The USB initialisation 
functions provided by the Nordic SDK will check if the Soft Device is enabled or 
not and set things up accordingly. The problem we experienced was a result of 
initialising USB first, then Bluetooth, which in turn enables the Soft Device. 
At the time USB was initialised, the Soft Device was not enabled so the SDK did 
not set up the hardware events to work with the Soft Device. Once the Soft Device 
was enabled by the Bluetooth initialisation, the hardware events would no longer 
work for USB. The result was the USB driver not knowing the device had been 
connected to a host, so it would remain idle and never make any attempt to 
enumerate or communicate with the host. While running in debug mode, there just
happened to be enough time for the hardware events to reach the USB driver after 
it was initialised but before Bluetooth was initialised. The solution to the problem
was to simply enable the Soft Device in the USB initialisation code before it
configures the hardware for USB detection. This way, when the SDK checks if the
Soft Device is enabled it will see that it is, and USB will be configured 
accordingly.
