Several challenges were faced during the course of this project with both the
hardware and software development. This section will describe some of the 
challenges faced, how they were overcome, and what trade-offs or compromises,
if any, had to be considered when finding a solution.

\subsection{Hardware Challenges}

One of the major challenges that occurred with the hardware development early
on in the project was the power supply. The original plan as described in the 
proposal was to use a single rechargeable AAA nickel metal hydride (NiMH) 
battery. Designing a power supply to work with the low input voltage of 1.2V 
to 1.0V proved to be quite difficult due to a small selection of components that 
work at that voltage. The most difficult part was trying to implement a soft 
power on / off circuit as well as an automatic switchover circuit to shift the 
electrical load from the battery to USB power. The voltage drop across a diode 
for example makes the resulting signal nearly unusable to turn on a MOSFET. 
Diodes with a very low forward voltage drop and MOSFETS with very low gate 
threshold voltages are available for a higher price, but keeping cost down was 
also a very important goal. In the end a design was created, and since it was 
expected that it may present some problems, it was prototyped on a breadboard 
so that testing could be done before moving ahead with schematic capture and 
PCB design.

Testing showed that the design mostly worked, but had some flaws that were 
deemed unacceptable for this project. The first concern was that the power 
supply would not be able to supply enough current. It was capable of powering 
an nRF52840 development board with some additional resistive load, but it 
struggled to maintain a set voltage and it became apparent that it would be 
barely capable of powering the final design with several sensors. The second 
issue was with voltage regulator start up under load. The switching regulator 
was only able to start with a very small load on it, but once the output had 
reached the desired voltage more load could be placed on it without issue. When 
too much load was attached to the regulator before start up it would stay in an 
under-voltage lockout state indefinitely. This problem may or may not have 
presented itself in a real-world test with the final design, however the goal 
was to have a robust solution that was reliable and not leave things to chance 
with a potential race condition. The final problem was efficiency. With a light 
load the power supply performed in an acceptable fashion, but under heavier 
loads efficiency dropped off much more than expected. The datasheet for the 
regulator did not indicate that efficiency would change appreciably over the 
range of current being drawn for testing, but it did indicate that lower input 
voltages would cause a significant decrease in efficiency. It is believed that 
higher load caused the battery voltage to drop due to its series resistance 
which then caused the regulator to draw much more current to account for the 
higher load and decreased efficiency at the same time because of the decreased 
input voltage. Replacing the battery with a lab power supply set to constant 
voltage showed the expected efficiency under both light and heavy loads. For 
these three reasons it was decided that the original idea of using a single 
rechargeable AAA NiMH battery was not feasible. 

At this point the group had to decide between using two AAA NiMH batteries in 
series to increase the voltage, or starting over and using a LiPo battery. It 
was decided that two AAA batteries would be too large, so the switch to LiPo 
was made and the power supply was subsequently redesigned to reflect these 
changes. One of the design requirements in the proposal was to have the 
batteries removable so they could be swapped out with charged ones instead of 
waiting for the device to charge them. This will no longer be possible with 
the current design; however fast charging capability has now been added as a 
replacement.

\subsection{Software Challenges}

Our toolchain setup ended up being much more difficult than we had envisioned.
In our proposal we had only budgeted ourselves a couple days for this step,
assuming that our experience with other ARM microcontrollers would allow us to
get up and running with the nRF52840 quite quickly. In reality, this seemingly
simple task had required a significant investment of time which held up our
software progress early in the project considerably.

%should we get rid of the \ref?
As described in our proposal (section \ref{subsec:proposed_software_libraries}),
we are used the hardware abstraction layer provided by Nordic
Semiconductor for the nRF52840 as well as some of their libraries. While we had
a build system set up that allowed us to successfully compile and run bare metal
software without any of Nordic's nRF SDK components, actually compiling the SDK 
along with our own code proved to be significantly more challenging. We 
originally tried to adopt our custom Makefile to also compile the nRF52 SDK files 
we required, but ended up having to base our project layout off of one of the 
example projects provided with the nRF52 SDK. A surprising amount of work was 
required to get the Makefile and Segger Embedded Studio project from the example 
to work in our repository without having to place all of our code in the examples 
folder inside the nRF52 SDK's directory structure.