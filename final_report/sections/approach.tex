This section describes our methodology and approach towards this project.

\subsection{Component Choices}

We have not yet chosen all of the components we plan to use, but we have done
significant research on the most important component choices and have selected
some of the key components for our system.

\subsubsection{Microcontroller}

We considered a number of requirements in choosing a microcontroller for our
project. We have identified several microcontroller features that are needed to
support our project requirements and did not consider any microcontrollers
without these features. These features are:
\begin{itemize}
    \item Bluetooth Low Energy support
    \item At least one I2C master mode interface for sensors
    \item An SPI or SD/MMC interface to connect and SD card
\end{itemize}

While we could have selected from a much wider variety of microcontrollers if we
had considered using a separate Bluetooth LE controller chip, we chose to limit
our selection to parts that featured an integrated Bluetooth LE radio. This is
to reduce part count (and therefore size, cost and design complexity) and to
provide better power consumption.

Table \ref{tab:mcu-comp-general} shows a brief summary of the microcontroller
options that we considered, all of which meet the minimum criteria outlined

above. The costs listed in the table are the price in Canadian dollars in single
quantities in the friendliest device package available and cut tape or tray
packaging from digikey.ca. These prices are often volatile and are therefore
listed here only for relative comparison.

All of the parts listed are based on ARM cortex processors. Nordic
Semiconductor's nRF5340 and the Silicon Labs EFR32BG22C224 use a Cortex-M33
while all of the other microcontrollers listed use a Cortex-M4. We estimate that
any of these microcontrollers would have sufficient processor performance for
our application. Similarly, all of the listed options have ample flash storage
and SRAM.

Not all of the listed microcontrollers come in packages that are easy to solder
by hand. The relatively small QFN packages of the nRF52832 and EFR32BG22C224 can
be soldered with relative ease using a hot air reflow station or an iron and be
inspected easily with a microscope. Unfortunately the other options are much
more difficult, the Apollo3 Blue's BGA package and the WLP package of the
MAX32668 each have a large array of solder balls. These packages can be soldered
with hot air reflow, but we do not have the equipment to inspect them. The
nRF52840 and nRF5340 are a middle ground, their aQFN packages have many fewer
balls, but they still do have two rows of connections. This means that while we
are able to solder the packages with hot air reflow they are difficult to
inspect. We may be able to sidestep the issue of soldering difficulty entirely
though, the nRF52840 and nRF52830 can be found on readily available modules that
integrate the microcontroller and a chip antenna on a small PCB. The Apollo3
Blue is available in a similar module from SparkFun branded as SparkFun Artemis.

The availability is also a limiting factor in our decision. Most of our listed
microcontrollers are readily available from a variety of online component
vendors. The exception is the Apollo3 Blue, which is not readily available
outside of SparkFun's Artemis module, and the MAX32668 which is a newly
available part and can currently only be ordered as a sample from Maxim's
website. While we could request samples of the MAX32668, there is no guarantee
that we would receive them, this is a serious downside to an otherwise very
good option.

\begin{table*}[htb]
\centering
\begin{tabular}{>{\centering\arraybackslash}m{2.2cm}|
                >{\centering\arraybackslash}m{2.5cm}|
                >{\centering\arraybackslash}m{2.0cm}|
                >{\centering\arraybackslash}m{1.5cm}|
                >{\centering\arraybackslash}m{1.2cm}|
                >{\centering\arraybackslash}m{1.8cm}|
                >{\centering\arraybackslash}m{1.2cm}}
\toprule
Part Number & Manufacturer & CPU & Program Memory & SRAM & Friendliest Package & Cost \\
\midrule
nRF52840 & Nordic Semiconductor & ARM Cortex-M4 at 64 MHz & 1 MiB & 256 KiB  & aQFN73 & \$9.34 \\
nRF52832 & Nordic Semiconductor & ARM Cortex-M4 at 64 MHz & 512 KiB & 64 KiB  & QFN48 & \$8.19 \\
nRF5340 & Nordic Semiconductor & Dual ARM Cortex-M33 at 96/64 MHz & 1 MiB & 512 KiB & aQFN94 & \$13.85 \\
Apollo3 Blue & Ambiq & ARM Cortex-M4 at 48 MHz & 1 MiB & 384 KiB & BGA81 & N/A \\
MAX32668 & Maxim Integrated & ARM Cortex-M4 at 96 MHz & 1 MiB & 560 KiB & WLP109 & N/A \\
EFR32BG22 C224 & Silicon Labs & ARM Cortex-M33 at 76.8 MHz & 512 KiB & 32 KiB & QFN40 & \$5.26 \\
\bottomrule
\end{tabular}
\caption{Comparison of Microcontrollers With Bluetooth Low Energy Radios}
\label{tab:mcu-comp-general}
\end{table*}

Table \ref{tab:mcu-comp-comm} compares the communications interfaces available
on the microcontrollers we considered. All of the micrcrocontrollers feature at
least Bluetooth Low Energy 5.0, at least two I2C interfaces and at least two SPI
interfaces. In addition to these interfaces the MAX32668 has an SD/MMC interface
controller which supports the SD3.0 specification. This interface would allow
significantly faster SD card access speeds than we can accomplish using an SPI
interface.

The nRF52840 and MAX32688 both feature USB interfaces. We plan to have a USB
CDC-ACM interface for debugging and as a fallback in case we can't get the
Bluetooth interface for data transfer to work. While we can implement a USB
interface with any of the microcontrollers listed by using a separate USB to
UART chip, using one of the microcontrollers with a built-in Bluetooth interface
would give us greater integration saving us cost, board space, complexity and
power usage.

\begin{table*}[htb]
\centering
\begin{tabular}{>{\centering\arraybackslash}m{3.0cm}|
                >{\centering\arraybackslash}m{1.8cm}|
                >{\centering\arraybackslash}m{1.8cm}|
                >{\centering\arraybackslash}m{1.8cm}|
                >{\centering\arraybackslash}m{1.5cm}|
                >{\centering\arraybackslash}m{3.0cm}}
\toprule
Part Number & I2C Master Interfaces & SPI Master Interfaces & SD/MMC Interface & BLE Version & USB \\
\midrule
nRF52840 & 2 & 4 & None & 5.2  & 2.0 Full Speed \\
nRF52832 & 2 & 3 & None & 5.2  & None \\
nRF5340 & 4 & 5 & None & 5.1 & None \\
Apollo3 Blue & 6 & 6 & None & 5.0 & None \\
MAX32668 & 3 & 3 & SD3.0 & 5.0 & 2.0 High Speed \\
EFR32BG22C224 & 2 & 2 & None & 5.2 & None \\
\bottomrule
\end{tabular}
\caption{Comparison of Microcontroller Communications Hardware}
\label{tab:mcu-comp-comm}
\end{table*}

Our application would benefit from hardware accelerated cryptography. Table
\ref{tab:mcu-comp-crypto} shows the cryptography features of the microcontrollers
that we considered. All of the microcontrollers have suitable hardware
accelerated cryptography features with the exception of the nRF52832 which lacks
any modular arithmetic acceleration for public key cryptography and the Apollo3
Blue which has no hardware cryptography accelerator.

\begin{table*}[htb]
\centering
\begin{tabular}{>{\centering\arraybackslash}m{2.5cm}|
                >{\centering\arraybackslash}m{2.0cm}|
                >{\centering\arraybackslash}m{2.0cm}|
                >{\centering\arraybackslash}m{2.5cm}|
                >{\centering\arraybackslash}m{2.5cm}|
                >{\centering\arraybackslash}m{1.5cm}}
\toprule
Part Number & Hashes & Public Key Algorithms & Symmetric Key Algorithms & AES Modes & True Random Number Generator \\
\midrule
nRF52840 & SHA-1, SHA-2 up to 256 bits, HMAC & RSA up to 2048 bit key, ECC & AES128, ChaCha20, Poly1305 & ECB, CBC, CMAC/CBC-MAC, CTR, CCM/CCM* & Yes \\
nRF52832 & None & None & AES128 & ECB, CCM & No \\
nRF5340 & SHA-1, SHA-1 up to 256 bits, HMAC & RSA up to 3072 bit key, ECC & AES128, AES256 & ECB, CBC, CMAC/CBC-MAC, CTR, CCM/CCM*, GCM & Yes \\
Apollo3 Blue & None & None & None & None & No \\
MAX32668 & SHA-2 up to 512 bits & RSA up to 4096 bit key, ECC & AES128, AES192, AES256, DES, 3DES & ECB, CBC, CFB, OFB, CTR & Yes \\
EFR32BG22 C224 & SHA-1, SHA-2 up to 256 bits & ECC & AES128, AES192, AES256 & ECB, CTR, CBC, CFB, GCM, CBC-MAC, GMAC, CCM & Yes \\
\bottomrule
\end{tabular}
\caption{Comparison of Microcontroller Cryptography Hardware}
\label{tab:mcu-comp-crypto}
\end{table*}

Because of its availability, relatively easy to work with package, suitable
serial interface, integrated USB support and hardware cryptography acceleration
we have chosen to use the nRF52840 microcontroller. The Apollo3 Blue has
tempting low power features, but questionable availability and a complete lack
of hardware accelerated cryptography. The MAX32668 would also have been a very
strong option, but it is not yet widely available. The EFR32BG33C224's lack of
an integrated USB interface and comparatively little documentation caused us to
choose the nRF52840 over it.

%\subsubsection{Optical Heart Rate and Pulse Oximetry Sensor}

%\todo{Sam: do this section}

%\subsubsection{Skin and Ambient Temperature Sensor}

%\todo{Sam: do this section}

%\subsubsection{Accelerometer}

%\todo{Sam: do this section}

\subsection{Hardware Tools}

We will require a number of hardware tools in order to accomplish our project.

\subsubsection{Microcontroller Development Boards}

In order to develop firmware for the nRF52840 microcontroller we will need
access to development boards.  These boards will allow us to begin work on the
firmware portions of our project before the hardware portions have been
completed and assembled.

The nRF52840-DK is well suited to our purposes, but is unfortunately also
relatively expensive.  While we already have one nRF52840-DK, we will need to
find other development board options so that all of our group members have
access to a platform for firmware testing.  To this end, we will make use of a
smaller development kit from Nordic Semiconductor, the nRF52840-DONGLE which is
a more inexpensive option.  Unlike the nRF52840-DK, this board does not have a
programer onboard. Instead it relies on a USB bootloader, which will make
debugging more difficult, but we imagine that it will be sufficient until we
have our own hardware ready.

\subsubsection{Microcontroller Programmer and Debugger}

In order to program and debug our final boards we will need to have programming
hardware that supports the SWD protocol and the nRF52840 microcontroller.  We
plan to use the Segger J-Link EDU mini programmer, which is an affordable
solution that supports a wide range of Cortex-M microcontrollers.  We already
have one Segger J-Link programmer and plan to acquire more so that all group
members can utilize one.

\subsubsection{Hardware Debugging Tools}

It is anticipated that we may also require hardware tools for debugging hardware
and firmware issues. We will make use of a logic analyzer and oscilloscope as
needed in order to help with hardware and firmware development. Both a Logic
Analyzer and an oscilloscope are available to us already and can be used for any
hardware or firmware debugging that requires these tools.

\subsection{Software Tools}

Several software tools are needed in order to complete our project.

\subsubsection{Toolchain and IDE}

In order to compile our code for the nRF52840 MCU we intend to use the Segger
Embedded Studio IDE with the arm-none-eabi-gcc toolchain. Segger Embedded Studio
is the vendor supported IDE for the nRF52840, and we expect its integration to
simplify working with the SDK and soft devices provided by Nordic Semiconductor.

\subsubsection{Libraries}
\label{subsec:proposed_software_libraries}

We plan to make use of the nRF5 SDK provided by Nordic Semiconductor which
includes a hardware abstraction layer for working with the nRF52840 SDK.  We
believe that this will save us considerable development time compared to working
without a hardware abstraction layer. While using an abstraction layer will
result in increased power consumption, we are willing to accept this trade-off
for this project because of our limited development timeline.

We also plan to use Nordic's Soft Device S113, which is a Bluetooth peripheral
stack, in order to save considerable time and effort in developing radio
firmware. It would likely be possible to produce a more power efficient
Bluetooth stack with a smaller footprint if we designed one that was specific to
our application. We feel that the development time saved by using the soft
device is worth these trade-offs.

\subsubsection{Schematic Capture and PCB Design}

For schematic capture and PCB design we plan to use KiCad.  KiCad is new to all
members of the project team, but since there is no single EDA program that
all group members are familiar with our plan is to learn a common
program.

\subsection{Mechanical Design}

The mechanical design of the watch will consist of a watch band and a watch
enclosure.  The watch band will be a commercial off the shelf component with a
standardized size.  The enclosure will be a 3D printed shell based around the
dimensions of the watchband and PCB.  It will be made up of two separate pieces,
a base that will sit against the users wrist, and a top which will face the
environment.  The base will feature protrusions on the sides that will contain
the watch band pins.  The base will also feature cutouts on the bottom for the
sensors.  The PCB will be screwed directly into the base.  The reason for
mounting the PCB to the base is to both maintain a regular distance from the
users skin and to allow PCB debugging when the device is affixed to a wrist.
The top of the enclosure will snap into the base.  The top will feature a grove
along the seam to aid in removing it from the base.

\subsection{Timeline}

Figure \ref{fig:project-timeline} shows our planned project timeline. A
description of each milestone in our timeline is given in section
\ref{subsubsec:milestones}.

\ganttset{
    calendar week text={\currentweek}
}

\begin{figure}[!htb]
\makebox[\textwidth][r]{
\begin{ganttchart}[
        y unit title=0.5cm, y unit chart=0.5cm,
        vgrid,
        x unit = 1mm,
        time slot format=isodate,
        time slot unit=day,
        title/.append style={draw=none, fill=RoyalBlue!50!black},
        title label font=\sffamily\scriptsize\color{white},
        title label node/.append style={below=-1.6ex},
        title left shift=.05,
        title right shift=-.05,
        title height=1,
        bar height=.6,
        group right shift=0,
        group top shift=.6,
        group height=.3,
        group peaks height=.2,
        bar incomplete/.append style={fill=Maroon},
        bar label node/.style={text width=3cm,align=right,font=\scriptsize\RaggedLeft,anchor=east},
        milestone label node/.style={text width=2cm,align=right,font=\scriptsize\RaggedLeft,anchor=east},
        group label node/.style={text width=3cm,align=right,font=\textbf\scriptsize\RaggedLeft,anchor=east}
    ]{2020-11-02}{2021-03-20}
    \gantttitlecalendar{year} \\
    \gantttitlecalendar{month} \\
    \gantttitlecalendar{week} \\
    
% Software
    \ganttgroup{Software}{2020-11-02}{2021-02-20} \\
    \ganttbar{Setup Dev. Env.}{2020-11-02}{2020-11-07} \\
    \ganttbar{USB Debugging Interface}{2020-11-08}{2020-12-19} \\
    \ganttbar{SD Card Driver}{2020-11-08}{2021-01-02} \\
    \ganttbar{Sensor Drivers}{2020-11-08}{2021-01-09} \\
    \ganttbar{Filesystem}{2020-12-12}{2021-01-16} \\
    \ganttbar{Bluetooth Interface}{2020-12-19}{2021-01-30} \\
    \ganttbar{Data Logging}{2020-12-26}{2021-01-30} \\
    \ganttbar{Companion Software}{2021-01-02}{2021-02-20} \\

% Testing
    \ganttbar{Sensor Driver Tests}{2021-01-09}{2021-02-13} \\

% Elec
    \ganttgroup{Hardware}{2020-11-02}{2021-03-06} \\
    \ganttbar{Electrical Schematic}{2020-11-02}{2020-12-12} \\
    \ganttbar{PCB Rev. A}{2020-12-13}{2021-01-30} \\
    \ganttbar{PCB Rev. B}{2021-01-31}{2021-03-06} \\

% Mech
    \ganttgroup{Enclosure}{2020-11-02}{2021-03-08} \\
    \ganttbar{Enclosure Prototype A}{2020-11-02}{2020-12-19} \\
    \ganttbar{Enclosure Prototype B}{2021-01-23}{2021-02-03} \\
    \ganttbar{Final Enclosure}{2021-02-27}{2021-03-08} \\
\end{ganttchart}
}
\caption{Project Timeline}
\label{fig:project-timeline}
\end{figure}

\FloatBarrier

\subsubsection{Milestones}
\label{subsubsec:milestones}

\paragraph{Development Environment Set Up - 2020-11-07}
A development environment will be set up for all members of
the group and should be able to compile code and program an nRF52840 development
board.

\paragraph{Electrical Schematic Complete - 2020-12-12}
A complete electrical schematic for our hardware will be developed and
reviewed and work on our PCB layout can commence.

\paragraph{Enclosure Prototype A - 2020-12-12}
A 3D printed prototype will be created.  This will be used to determine PCB
dimensions that are both reasonable and comfortable, to validate the snap
mechanism for holding the two piece enclosure together, and to validate the
wrist band mechanism.

\paragraph{USB Debugging Interface Complete - 2020-12-19}
Software that allows for interactive testing on the microcontroller via a
USB-CDC interface will be completed.

\paragraph{PCB Revision A Ordered - 2021-01-02}
Complete the PCB layout and place order for manufacturing.

\paragraph{SD Card Driver Complete - 2021-01-02}
The driver software that will allow us to
read and write blocks on a micro SD card over our microcontroller's SPI
interface will be completed.

\paragraph{Sensor Drivers Complete - 2021-01-09}
Completed and performed basic functional testing for the
sensor drivers required for our system.  This includes the drivers for the
temperature sensors, optical heart rate/pulse oximetry sensor and the IMU.

\paragraph{Filesystems Code Complete - 2021-01-16}
Completed adapting the filesystem code for our
application and have the ability to read and write files on an SD card.

\paragraph{PCB Revision A Assembled - 2021-01-30}
First revision of assembled PCBs and first firmware trial run on the board.
Basic electrical and firmware testing completed on the board and at this 
point, any issues that would require a new PCB revision to be made will be
determined.

\paragraph{Data Logging Code Complete - 2021-01-30}
Complete code for logging sensor data onto the SD
card and reading it back to be transmitted over bluetooth.

\paragraph{Bluetooth Interface Code Complete - 2021-01-30}
Code that allows for communication between a simple test
application and the microcontroller over Bluetooth completed.

\paragraph{Enclosure Prototype B - 2021-02-03}
Using PCB Revision A, an enclosure will be fabricated to validate sensor
positions and the PCB mounting mechanism.

\paragraph{PCB Revision B Ordered (if needed) - 2021-02-06}
If there are any issues found with revision A of our PCB design, changes will
be made and the second revision will then be ordered.

\paragraph{Sensor Driver Tests Complete - 2021-02-13}
Complete unit and function tests written for the sensor driver code.  The sensor
drivers should be fully tested and any problems discovered during testing
should have been fixed.

\paragraph{Companion Software Complete - 2021-02-20}
Companion software is complete and sensor data can be downloaded to the 
companion software via Bluetooth.

\paragraph{PCB Revision B Assembled (if needed) - 2021-03-06}
Completed assembly of the second revision of our PCB design, if required.
Full firmware is running on the second revision board.

\paragraph{Final Enclosure - 2021-03-08}
Using PCB Revision B, the final enclosures will be fabricated.  This will be the
enclosure the watch is demonstrated in.
%It will be printed in a variety of
%colours and painted with beautiful depictions of an archer fish \cite{archerFish}.

%\todo{Morgan: Change fish name to Fred}

\subsection{Risks and Mitigation}

We have identified several areas in our project that are relatively high risk and
have taken steps to mitigate them.

\subsubsection{PCB Manufacturing Lead Times}

One of the most significant areas of risk for our project is the hardware
design. Inflexibility, long lead times for components and PCBs, high cost and
more difficult debugging make the hardware aspect come with many risks. Having
anticipated several potential pitfalls in our hardware development process, we
have incorporated into our approach a number of aspects designed to mitigate
them.

One of the largest potential sources of delays and uncertainty is the long lead
times for PCB manufacturing. Ordering PCBs that are manufactured in North
America is often significantly more expensive, so we plan to order our PCBs from
a Chinese manufacturer. This means that on top of the time required to build the
boards there will be a fairly substantial shipping delay. This makes it very
difficult to iteratively design a circuit board over our relatively short
time frame.

In order to mitigate this, we aim to work on our hardware design quickly and
early on to allow us to order a first revision of our PCB as soon as possible. This
will ensure that we have time to thoroughly test it and leave us enough time to
order a second revision of the PCB if required. The Chinese New Year in 2021
begins on February 12th, at which point the manufacturing facilities in China
will be shut down for at least a week. In order to maintain our project time
line we aim to order the second revision of our PCB, if required, at least one
week prior to the start of the New Year so that we will not be delayed by the
holiday. If for some reason we require boards with a quick turnaround we can
order them from a US manufacturer such as OSH Park, but that will likely be
significantly more expensive.

\subsubsection{PCB Assembly}

Difficulty in manufacturing PCBs is another potential source of risk.  While we
have equipment for surface mount soldering, we lack any kind of reflow oven or
advanced inspection equipment.  Because of this, there is a limit to the kinds
of component packages that we can make use of in our design.  For instance large
BGA packages would be very difficult for us to solder reliably and impossible
for us to inspect.

We plan to mitigate this risk in our board design by planning out our component
choices with the aforementioned restriction in mind, making sure to choose
components that will be within our assembly capabilities.  If required, we also
have the option of potentially making use of some more advanced PCB assembly
equipment that is available to us through external contacts.

\subsection{Test Plan}

We plan to write automated or semi-automated tests for each component of our
project. Most of these tests will take the form of unit tests, test harnesses
that emulate adjacent modules or small test applications that exercise a single
module.

\subsubsection{Unit Tests}

We plan to build a simple unit testing framework that allows us to write unit
tests that can be compiled and run on a host system. These tests will target
individual functions and will aim to validate that they are correct.

It is unlikely that we will be able to achieve full test coverage due to time
constraints. We will therefore focus on unit testing areas of the code that
are difficult to write test harnesses for, such as sensor drivers and the SD
card driver.

\subsubsection{Sensor Driver Testing}

In addition to unit tests, we will test the sensor drivers with semi-automated
test applications that run on the microcontroller. These tests will provide
feedback over the USB-CDC interface and will interact with the actual sensor
hardware.  Our sensor tests will aim to exercise all of the functionality of the
sensors that our application code makes use of.

These tests will be in the form of small console applications that will take
a series of sensor readings and display them. These readings will be manually
verified, since automatic verification of output that is based on real world
inputs is difficult. Due to the difficulty in writing automated tests for the
sensor drivers, these drivers will be an area of focus for unit testing.

\subsubsection{Bluetooth Stack Testing}

Our Bluetooth related firmware components will be developed alongside the
companion software. As they are developed together we will test them together,
using each to verify the correct operation of the other. Much of this testing
will be performed manually as we anticipate that tests that require interaction
between our device and the companion software will be more difficult to
automate.

\subsubsection{SD Card Driver Testing}

We will test the SD card driver and filesystem software using test applications
that will run directly on the microcontroller. A test application that reads
and writes blocks on the SD card and a test application that is able to read
from and write to files will be developed in order to validate the functionality
of the SD card and filesystem code. These tests will be small console
applications that a user can interact with using the the USB-CDC interface.

\subsubsection{Data Logging Code Testing}

In order to test the data logging component of the project we will create a test
framework that runs on a desktop computer which will emulate the sensor input
and the filesystem components in order to ensure that the application layer
encryption and file packing design works as intended. This will allow the data
logging code to be developed and tested independently of the filesystem and SD
card drivers to allow for more parallel development. This test framework will
create snapshots of the outputted data for a set of input conditions and ensure
that any updates to the logging code either does not change the output to ensure
nothing is broken during development, or if there is an error in these snapshots
then the snapshots will be updated.

