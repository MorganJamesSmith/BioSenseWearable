This section will discuss both the criteria for success and what measures will be used
to meet these criteria.

The purpose of this project is to design and build a wrist mounted wearable
health monitoring device.  The device will provide a sensor platform that is
suitable for early detection of disease.  The sensors chosen will measure skin
and ambient temperature, heart rate, pulse oximetry, and motion.  The electronic
components will fit into a comfortable, wearable 3D printed plastic enclosure.
The devices will be powered by a AAA battery to ensure maintainability.  Since
the battery is the largest single component of the device, the device should be
roughly the width of a wrist and
% Rectangular device - not explicitly mentioning it's rectangular.
the length of the AAA battery plus the battery contacts.  The device will also
have a charging circuit to allow rechargeable AAA batteries to be charged with
a USB connector.

The bulk of the electronics will consist of a microcontroller that supports
Bluetooth for ease of extracting logged data off of the device and a USB
connection for development and charging.  A microcontroller of relatively low
cost will be chosen if multiple options with the above requirements are
available.

% - Exposed serial interface for end user expansion

The wearable device will have limited user interfaces consisting of a indicator
LED for power, notifications and debugging and a button for power.
% - Blinky LED and USB serial interface for debugging
The main interactions with the device will occur through a companion software
which will extract recorded data from the device and display it as graphs or
export in a analyzable format such as a csv or json file.  This companion
software will connect to the device through Bluetooth or USB to extract data.

The device will store the collected sensor information to a micro SD card
in the device which is then read back when requested via the companion software.
The SD card will be encrypted so that unauthorized access to the data will be
more difficult, maintaining the privacy of the user's biometrics. This
encryption will likely happen in the application layer as well as the bluetooth
transport layer so that physical access to the device does not allow the data to
be read without proper authority. 
The SD card used will be replaceable, the capacity of the card used will dictate
how long the device can record data before needing to off load to the companion
software and this size will be decided by the user.

The overall cost of the device will be minimized while preserving all functional
requirements, particularly by using readily available electronic components to
ensure the device is easy to produce without additional effort.

The focus of the project will be on collection of biometrics and making them
available for analysis. We do not plan to attempt to perform medical analysis
or give clinical diagnosis from the collected biometrics. Our intention is to
design the system so that those applications could be integrated with or built
on top of the work we will do in this project.

The companion software will be limited to simply receiving, decoding, and
decrypting data received from the wearable device. We may integrate some basic
data visualization either in the companion app or as a separate application, but
this will be solely for demonstration purposes and will not be considered part
of the project requirements.

In the interest of reducing development cost and complexity it was decided to
use sensors that are readily available from online sources, and that can be
assembled on a PCB by hand relatively easily.  The selection of sensors may be
more limited or lower quality than what would have been possible if options
that are made to order, or only available in large quantities had been
explored.  The use of parts that can be soldered to a PCB without paying for
the services of an advanced manufacturing facility is also likely going to
result in the final hardware design being larger than it could be.
