This report details the design and creation of a continuous biometric measurement
device costing a total of \costofboard. This device takes measurements from the user
and prepares them for analysis to detect illness or otherwise track well being.

Traditionally, people only seek treatment for ailments after they start
experiencing symptoms; however, it is often desirable to take action before
symptoms appear.  One solution to catch illness earlier could be to monitor a
person's physiological state using various sensors and take action when the
state falls outside of a known healthy state.

Wearables are capable of revealing personal daily physiological cycles with
respect to skin temperature, and heart rate with high accuracy 
\cite{Schuurmans2020}. By comparing a person's current physiological
state with the expected state, it is possible to predict illness with greater
sensitivity than a self diagnosis \cite{POON2014543}. 
One specific example is a study was able to detect a
period of inflammation (confirmed by a medical high-sensitivity C-reactive
protein test) using a participant's heart rate even though the participant
themselves experienced no symptoms \cite{Wearable-tracking2017}.  The same
study even helped a participant get treatment of Lyme disease before the
``Bull's eye'' rash appeared (a common symptom during the early stages of Lyme
disease) by detecting an abnormality in their SpO\textsubscript{2} levels.

Wearable data can also be used to detect group illness rates.  By using heart
rate data and sleep data from fitbit devices and cross referencing that data
with the Centers for Disease Control and Prevention's (CDC) Influenza Like
Illness (ILI) data, a study was able to predict illness 6.3\% to 32.9\% better
than previous methods \cite{Radin2020}.

A survey on health sensing wearables showed that physiological data collected 
over longer periods of time can be useful for gait analysis and activity 
quantification, among other things \cite{Song2014}. Gait analysis is useful for 
patient mobility evaluation in hospitals, can help diagnose some diseases, and 
can also be used as a more general indicator of health \cite{Song2014}. Activity
quantification is import to aid patients undergoing rehabilitation \cite{Song2014}.

Wearables can also detect more immediate dangers to someone's health like a
heart attack \cite{heart-attack}, a fall \cite{Khojasteh_2018}, or dangerously
low or high heart rates.  This information could be used to alert emergency
services to save lives.

In addition to collecting physiological data, many wearables can also measure
environmental data and positional data \cite{Seneviratne2017}. This data has 
proven useful in dangerous work environments such as those commonly 
found in the mining industry \cite{Mardonova2018}. Air quality information can
help miners avoid inhaling hazardous gasses or respirable silica dust 
\cite{Mardonova2018}. Some wearables will even detect when equipment operators
are becoming sleepy and warn them that it is no longer safe to continue working
\cite{Mardonova2018}. In the construction industry, there is a high rate of 
severe contact injuries \cite{Awolusi2018}. The use of location data can help 
reduce these contact injuries by preventing workers from entering dangerous 
areas or coming into contact with equipment \cite{Awolusi2018}. Wearable devices
capable of measuring both physiological data and elevation can be used to detect
altitude sickness, which can occur when unacclimated individuals experience sudden 
changes in elevation \cite{Muza2018}. These sudden changes in elevation are 
possible while driving, riding a train, flying, or taking a cable car up a mountain 
\cite{Muza2018}.

\subsection{Motivation}

Despite all of the advances in medical technology, a cure for diseases and
allowing humans to live longer are ever present goals for scientists and 
doctors alike.  Often wide spread diseases we have not prepared for can occur in
society.  COVID-19 is a pandemic that
has, at the time of writing, claimed an estimated 2.5 million lives globally
\cite{johns-hopkins-corona-chan}, and continues to infect more people every
day.  In 2007 it was estimated that the total cost of occupational illnesses
among civilians in the United States was \$58 billion \cite{Leigh2011}.
Slowing the spread of disease could save lives and save money.

One method of limiting the spread of disease is early diagnosis.  This would
allow individuals who are aware they are ill to take measures to reduce the
spread of their illness, even before symptoms manifest.  This will be
particularly useful for healthcare providers who work with high risk
individuals, since identifying symptoms early has the highest efficiency for
protecting others.

By creating a personal physiological profile and constantly
comparing the user's current physiological state to it, illness can be detected
early.  The motivation for our project is to create a device capable of measuring 
enough of the user's current physiological state to predict illness.  This is done 
in the hope that it could be used to slow the spread of illness's and, to speed
recovery from illness's by starting treatment earlier.

\subsection{Problem Statement}

One way to facilitate early diagnosis is to periodically measure biometrics on 
patients and make clinical diagnoses based on patterns or changes in the 
measurements. A convenient and low-cost method to collect and log biometric data 
is needed in order to make the potential benefits of the technology available to 
a large number of people. Wearable devices are well suited to this application 
because they have continuous access to the user’s biometric data as long as the 
device is being worn, and they do not require any input from the user to perform 
most, if not all, measurements. This means that a wearable device can take as 
many measurements as it needs, whenever it needs to, in order to create useful 
data logs that describe the user’s biometrics and perhaps their environment at 
the time the measurements were taken.  This information may then be used to 
detect abnormalities in the user’s physiology and could be used to aid in the 
early detection of illness. The objective of this project therefore, is to 
create a wearable device that can measure a variety of critical health metrics 
including skin temperature and heart rate.

\subsection{Solution}

The solution provided by this project is a wrist mounted wearable health 
monitoring device. The device provides a sensor platform that is suitable for 
early detection of disease. The included sensors can measure skin temperature, 
ambient temperature, motion, heart rate, pulse oximetry, UV light, and air quality.  
The electronic components fit into a comfortable, wearable 3D printed plastic enclosure.
The device is powered by a lithium polymer battery which can be recharged
through the USB-C connector.  The final cost of the full device ends up being \costofboard.

The wearable device has limited user interfaces consisting of an indicator
LED for power, notifications, and debugging. It also features a power button
to turn the device on and off. The main interactions with the device occur 
through a companion software that connects to the device through Bluetooth or 
USB. It is limited to simply receiving and decoding data received from the 
wearable device, and exporting it in an analyzable format such as a csv or json 
file.

The device stores the collected sensor information to a micro SD card
in the device which is then read back when requested via the companion software.
The SD card is replaceable, and the capacity of the card used will dictate
how long the device can record data before needing to off load to the companion
software.

The focus of this solution is on the collection of biometrics and making them
available for analysis. No attempt was made to perform medical analysis
or give clinical diagnosis from the collected biometrics. Our intention was to
design the system so that those applications could be integrated with or built
on top of the work we have done for this project.

%Should this be in a reflection or something? Not really part of a solution,
%it is a disclaimer.

%In order to reduce development cost and complexity, the hardware component of
%the project uses sensors that are readily available from online sources, and 
%that can be assembled on a PCB by hand relatively easily. As a result the 
%selection of sensors may be more limited or lower quality than what is possible 
%if options that are made to order, or only available in large quantities had been
%explored.  The use of parts that can be soldered to a PCB without paying for
%the services of an advanced manufacturing facility also resulted in the final 
%hardware design being larger than it would have been otherwise.

\subsection{Accomplishments}

Our team successfully created a wrist mounted wearable health monitoring device
that is capable of measuring skin temperature, ambient temperature, and motion.
The hardware for heart rate, pulse oximetry, and UV light sensors are in place
however the software drivers were not completed in time. 
The air quality sensor was never available so the driver was not started and was
never populated on the PCB. The hardware is fully functional and correctly
connects all the selected components together.  The software is able to connect
the device to companion software located on a separate device using Bluetooth
and USB.  We are able to log motion and temperature sensor to the SD Card.

\subsection{Report Structure}
Subsequent sections of this
report cover the hardware and software design of the project in detail, as well
as discuss some of the challenges that were faced in pursuing the solution
outlined here. Information regarding the tests performed on the final project
are also detailed in this report, and a series of appendices are included with
visuals to show some interesting parts of the design and what the final product
looks like.
