This section contains reflections for each group member. These reflections
discuss how each group member felt the project went, and what they would do
different if they had to do it over again.

\subsection{Sam}

From my perspective I think that our project went fairly well, but that we
probably should have limited our scope further. Our hardware design and assembly
went relatively smoothly. There where a couple minor issues that could have been
fixed if our timeline allowed for a second PCB revision and we ended up having
to leave off some parts (such as the BME680 air quality sensor) and find
substitutions for others due to availability issues, but in general I am happy
with how our hardware turned out. Our software did not turn out as well, we
where slow to get started with the proper toolchain and SDK for the
microcontroller we choice to use and ended up spending much of our development
time learning to use the nRF5 SDK rather than writing our own application
software.

I think that we could have made a stronger push into software development
earlier, which may have helped with some of the software issues we ended up
with. We could also have prioritized choosing a microcontroller that we where
more familiar with over picking one with integrated Bluetooth. Instead, we could
have used a separate Bluetooth module or relied solely on USB for data transfer.
Using a microcontroller that we had previous experience with would likely have
allowed us to move much more quickly on software development.

I also think that we should have foreseen the issues we had with obtaining the
components that we used in our design, which where likely related to the widely
reported electronics supply chain shortages due to the ongoing pandemic. We
initially ordered only enough of most of the components we used for a single
copy of the board since we where worried about wasting our budget if it turned
out that our board design had problems and need to be changed before we
assembled the second copy. In retrospect we probably didn't have time for a new
PCB revision anyways, and even if we had designed a second revision it is very
unlikely that we would have made changes as large as replacing any of the major
components. We should have ordered enough components for two copies of the
board right away to ensure that we purchased all of the parts we needed when
they where available.

\subsection{Tadhg}

Reflection.

\subsection{Jason}

Overall, I think our project went ok, especially considering how bad it could
have gone. I felt that the project was risky from the start because a large
component of it is hardware design. There was very little room for error in the
board design because we only had time for one revision, manufacturing times
and component availability were out of our control, and simple mistakes in board 
design or assembly could have damaged many components wasting time and money. 
The COVID situation only made things more complex because we lost some of our 
backup plans for board assembly and could not easily work together on hardware 
development, debugging, and testing. All of that being said, I think we did a 
very good job of adapting and overcoming the issues we faced despite the cards 
not being in our favour. I was aware of all of these potential problems going
into the project, but I enjoy hardware development and was confident in our 
abilities so I was willing to take on the risk. What I did not foresee was the
software component of the project lagging behind as much as it did. I had no
worries about the software development at the beginning of the project 
because I thought it was very low risk, but by the end of the project it was
clear that I was wrong. I was not as involved with software development as I was
with hardware development so I am not the best person to say what we should 
have done different, but I think using a microcontroller that at least one of us
was familiar with would have helped. One thing we could have done differently to 
make hardware development much easier would be to make an entirely different 
project that does not depend on obscure sensors. Many of the sensors used in the 
project have little to no alternatives, and they tend to come in small packages 
that are hard to work with. We also could have made it easier by making 
something where size is not as relevant as it is in a watch.

\subsection{Morgan}

Reflection.
